\documentclass[12pt,a4paper]{article}

\usepackage[utf8]{inputenc}
\usepackage{geometry}
\usepackage{graphicx}
\usepackage{amsmath}
\usepackage{hyperref}
\usepackage{caption}
\captionsetup{hypcap=true}
\usepackage{float}
\usepackage{natbib}   % For name-year citations
\usepackage{url}      % For website URLs
\geometry{margin=2.5cm}
\usepackage{pgfplots}
\usepackage{pgfplotstable}
\pgfplotsset{compat=1.18}
\usepackage{subcaption}
%==================================================================
\title{From Solar Wind to Auroras: Magnetosphere--Ionosphere Response During the May 2024 Geomagnetic Storm}
\author{Stan Daniels\\
University of Oslo\\
Space Physics and Technology (FYS3600)}
\date{november 2025}
%==================================================================
\begin{document}
\maketitle
\begin{abstract}
\textbf{Add abstract here}
\end{abstract}
\tableofcontents
\newpage
%==================================================================
\section{Introduction}
Geomagnetic storms are disturbances in the Earth's magnetosphere caused by enhanced solar wind--magnetosphere coupling, which can lead to substantial variations in the geomagnetic field.
The theoretical framework for magnetic storms was first formalized by Chapman and Ferraro \cite{chapman1933new}, who described the characteristic phases of storm development.
Historically, geomagnetic storms were associated with auroral phenomena and disruptions in early telecommunication systems.
In contemporary contexts, they can induce geomagnetically induced currents (GICs) in power grids, degrade satellite operations, and compromise high-precision navigation systems.
Notably, strong geomagnetic storms have been shown to significantly affect high-latitude Real-Time Kinematic (RTK) positioning networks, resulting in reduced accuracy of precise geolocation measurements \cite{jacobsen2012observed}.
Investigating the mechanisms and impacts of geomagnetic storms is essential for both advancing space weather science and mitigating risks to technological infrastructure.\\\\
% white line
This study utilizes one of the most intense geomagnetic storms of recent times \cite{tulasi2024super}, to demonstrate how changes in the solar wind towards Earth modulate the magnetosphere-ionosphere (M-I) system. 
Several datasets have been used to investigate these effects.  
The \textit{OMNI dataset} \citep{OMNI2025} provides high-resolution solar wind and interplanetary magnetic field measurements, including the magnetic field components (\(B_x\), \(B_y\), \(B_z\)), total field magnitude (\(B\)), geomagnetic indices (e.g., SYM-H), and plasma parameters such as flow speed, proton density, and dynamic pressure, with time parameters shifted to the nose of the Earth's bow shock.  
The \textit{AMPERE dataset} \citep{Anderson2000, AMPEREwebsite} provides global field-aligned current observations derived from the Iridium satellite constellation, revealing how solar wind energy and momentum are transmitted into the ionosphere.  
Additional datasets may be included following the same structure: specify the source, measured variables, temporal and spatial resolution, and any preprocessing (e.g., coordinate transformations or time-shifting).  
This approach ensures consistency and clarity when describing multiple instruments and data sources throughout the report.
%==================================================================
\section{Observations}
\begin{figure}[H]
    \centering

    % Bz
    \begin{subfigure}[b]{\textwidth}
        \includegraphics[width=\textwidth]{../Processed_Data/OMNI_Data/Bz.png}
        \caption{Bz: southward magnetic field.}
        \label{fig:OMNI1_Bz}
    \end{subfigure}

    \vspace{0.3cm}

    % B magnitude
    \begin{subfigure}[b]{\textwidth}
        \includegraphics[width=\textwidth]{../Processed_Data/OMNI_Data/B_magnitude.png}
        \caption{B magnitude: total magnetic field strength.}
        \label{fig:OMNI1_Bmag}
    \end{subfigure}

    \vspace{0.3cm}

    % SYM-H index
    \begin{subfigure}[b]{\textwidth}
        \includegraphics[width=\textwidth]{../Processed_Data/OMNI_Data/SYM_H_index.png}
        \caption{SYM-H index: ring current intensity.}
        \label{fig:OMNI1_SYM-H}
    \end{subfigure}

    \caption{OMNI 1-minute data for 10–11 May 2024 geomagnetic storm: (a) $B_z$ component showing southward turning at 17:30~UT, (b) total IMF magnitude $B$, and (c) SYM-H index indicating the storm main phase.}
    \label{fig:OMNI1}
\end{figure}


\begin{figure}[H]
    \centering

    % Flow speed
    \begin{subfigure}[b]{\textwidth}
        \includegraphics[width=\textwidth]{../Processed_Data/OMNI_Data/flow_speed.png}
        \caption{Flow speed}
    \end{subfigure}

    \vspace{0.3cm}

    % Proton density
    \begin{subfigure}[b]{\textwidth}
        \includegraphics[width=\textwidth]{../Processed_Data/OMNI_Data/proton_density.png}
        \caption{Proton density}
    \end{subfigure}

    \vspace{0.3cm}

    % Flow pressure
    \begin{subfigure}[b]{\textwidth}
        \includegraphics[width=\textwidth]{../Processed_Data/OMNI_Data/flow_pressure.png}
        \caption{Dynamic pressure}
    \end{subfigure}

    \vspace{0.3cm}

    \caption{OMNI 1-minute data: magnetic field components, B magnitude, solar wind speed, proton density, dynamic pressure, and SYM-H index.}
    \label{fig:OMNI_all_vertical}
\end{figure}


\begin{figure}[H]
    \centering
    \includegraphics[width=0.35\textwidth]{../Processed_Data/AMPERE/AMPERE_20240510_0900.png}
    \hspace{0.05\textwidth}
    \includegraphics[width=0.35\textwidth]{../Processed_Data/AMPERE/AMPERE_20240510_1200.png}
    \caption{AMPERE Northern Hemisphere FAC maps showing the conditions before the 10–11 May 2024 geomagnetic storm. 
    The maps at 2024-05-10 09:00 and 2024-05-10 12:00 UT show weak upward (red) and downward (blue) currents.}
    \label{fig:Ampere1}
\end{figure}
\begin{figure}[H]
    \centering
    \includegraphics[width=0.35\textwidth]{../Processed_Data/AMPERE/AMPERE_20240510_1800.png}
    \hspace{0.05\textwidth}
    \includegraphics[width=0.35\textwidth]{../Processed_Data/AMPERE/AMPERE_20240511_0200.png}
    \caption{AMPERE Northern Hemisphere FAC maps showing the later stages of the 10–11 May 2024 geomagnetic storm. 
    The maps at 18:00 UT and 02:00 UT show strong upward (red) and downward (blue) currents. 
    These saturated colors indicate that region 1 and region 2 field-aligned currents have intensified, 
    reflecting peak FAC activity, an expanded polar cap, and enhanced auroral electrojets during the main and early recovery phases of the storm.}
    \label{fig:Ampere2}
\end{figure}
%==================================================================
\section{Discussion}
%==================================================================
\section{Conclusion}
%==================================================================
\bibliographystyle{apalike}  % Or another name-year style
\bibliography{references}    % 'references' is the name of your .bib file (without .bib extension)
%==================================================================
\end{document}
