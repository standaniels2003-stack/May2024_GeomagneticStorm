\documentclass[12pt,a4paper]{article}

\usepackage[utf8]{inputenc}
\usepackage{geometry}
\usepackage{graphicx}
\usepackage{amsmath}
\usepackage{hyperref}
\usepackage{caption}
\captionsetup{hypcap=true}
\usepackage{float}
\usepackage{natbib}   % For name-year citations
\usepackage{url}      % For website URLs
\geometry{margin=2.5cm}
%==================================================================
\title{From Solar Wind to Auroras: Magnetosphere--Ionosphere Response During the May 2024 Geomagnetic Storm}
\author{Stan Daniels\\
University of Oslo\\
Space Physics and Technology (FYS3600)}
\date{november 2025}
%==================================================================
\begin{document}
\maketitle
\begin{abstract}
\textbf{Add abstract here}
\end{abstract}
\tableofcontents
\newpage
%==================================================================
\section{Introduction}
Geomagnetic storms are disturbances in the Earth's magnetosphere caused by enhanced solar wind--magnetosphere coupling, which can lead to substantial variations in the geomagnetic field.
The theoretical framework for magnetic storms was first formalized by Chapman and Ferraro \cite{chapman1933new}, who described the characteristic phases of storm development.
Historically, geomagnetic storms were associated with auroral phenomena and disruptions in early telecommunication systems.
In contemporary contexts, they can induce geomagnetically induced currents (GICs) in power grids, degrade satellite operations, and compromise high-precision navigation systems.
Notably, strong geomagnetic storms have been shown to significantly affect high-latitude Real-Time Kinematic (RTK) positioning networks, resulting in reduced accuracy of precise geolocation measurements \cite{jacobsen2012observed}.
Investigating the mechanisms and impacts of geomagnetic storms is essential for both advancing space weather science and mitigating risks to technological infrastructure.\\\\
In this study one of the most intense geomagnetic storm of recent times\cite{tulasi2024super} is used to demonstrate how changes in the solar wind towards earth end up modulating the magnetosphere-ionosphere (M-I) system.
%==================================================================
\section{Observations}
The evolution of the 10–11 May 2024 geomagnetic storm is illustrated using AMPERE Northern Hemisphere field-aligned current (FAC) maps \citep{Anderson2000, AMPEREwebsite}.
Figure~\ref{fig:ampere1} shows the early stages of the storm.
At 09:00 UT, FACs are weak, with pale red and blue colors indicating minimal upward and downward currents, reflecting pre-storm conditions with limited dayside reconnection and a small polar cap.
By 12:00 UT, the currents begin to intensify, signaling the onset of enhanced region 1 and region 2 currents as solar wind-magnetosphere coupling strengthens.
Figure~\ref{fig:ampere2} presents the later stages at 18:00 UT and 02:00 UT.
The FACs are now strongly developed, with saturated red and blue colors representing intense upward and downward currents.
This indicates the storm main phase, with expanded region 1 and 2 currents, a larger polar cap, and enhanced auroral electrojets.
The progression from weak to strong currents demonstrates the gradual response of the magnetosphere-ionosphere system to changing solar wind conditions, as captured by AMPERE observations \citep{Anderson2000, AMPEREwebsite}.
\begin{figure}[H]
    \centering
    \includegraphics[width=0.35\textwidth]{../Processed_Data/AMPERE/AMPERE_20240510_0900.png}
    \hspace{0.05\textwidth}
    \includegraphics[width=0.35\textwidth]{../Processed_Data/AMPERE/AMPERE_20240510_1200.png}
    \caption{AMPERE Northern Hemisphere FAC maps showing the conditions before the 10–11 May 2024 geomagnetic storm. 
    The maps at 2024-05-10 09:00 and 2024-05-10 12:00 UT show weak upward (red) and downward (blue) currents.}
    \label{fig:Ampere1}
\end{figure}
\begin{figure}[H]
    \centering
    \includegraphics[width=0.35\textwidth]{../Processed_Data/AMPERE/AMPERE_20240510_1800.png}
    \hspace{0.05\textwidth}
    \includegraphics[width=0.35\textwidth]{../Processed_Data/AMPERE/AMPERE_20240511_0200.png}
    \caption{AMPERE Northern Hemisphere FAC maps showing the later stages of the 10–11 May 2024 geomagnetic storm. 
    The maps at 18:00 UT and 02:00 UT show strong upward (red) and downward (blue) currents. 
    These saturated colors indicate that region 1 and region 2 field-aligned currents have intensified, 
    reflecting peak FAC activity, an expanded polar cap, and enhanced auroral electrojets during the main and early recovery phases of the storm.}
    \label{fig:Ampere2}
\end{figure}
%==================================================================
\section{Discussion}
%==================================================================
\section{Conclusion}
%==================================================================
\bibliographystyle{apalike}  % Or another name-year style
\bibliography{references}    % 'references' is the name of your .bib file (without .bib extension)
%==================================================================
\end{document}
